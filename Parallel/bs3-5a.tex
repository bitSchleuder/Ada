%%%%%%%%%%%%%%%%%%%%%%%%%%%%%%%%%%%%%%%%%%%%%%%%%%%%%%%%%%%%%%%%%%%%%%%
% BS3.TEX - 3.Abschnitt der Vorlesung Betriebssysteme 
% Unterabschnitt 3.5
% Stand 20.05.2000
%%%%%%%%%%%%%%%%%%%%%%%%%%%%%%%%%%%%%%%%%%%%%%%%%%%%%%%%%%%%%%%%%%%%%%%
\typeout {Realisierung des "uberlappten Stapelbetriebes}
%%%%%%%%%%%%%%%%%%%%%%%%%%%%%%%%%%%%%%%%%%%%%%%%%%%%%%%%%%%%%%%%%%%%%%%
\subsection {Realisierung des "uberlappten Stapelbetriebes}
%%%%%%%%%%%%%%%%%%%%%%%%%%%%%%%%%%%%%%%%%%%%%%%%%%%%%%%%%%%%%%%%%%%%%%%

Durch die neugewonnene Interruptf"ahigkeit 
k"onnen wir uns bei unserem Stapelsystem 
von der rein sequentiellen Arbeitsweise befreien 
und sind nun in der Lage, 
den effizienteren "uberlappenden Betrieb durchzuf"uhren.

%%%%%%%%%%%%%%%%%%%%%%%%%%%%%%%%%%%%%%%%%%%%%%%%%%%%%%%%%%%%%%%%%%%%%%%

Um die "Uberlappung auszunutzen, 
holen wir die Kartenbilder bei der Eingabe 
nicht direkt vom Kartenleser ab, 
sondern aus einem Pufferbereich,
der parallel dazu nachgef"ullt wird. 
Auf der Ausgabeseite verfahren wir ebenso;
wir legen die produzierten Ausgabezeilen 
in einen Druckpuffer ab und erwarten, 
da"s sie dann parallel zur weiteren Verarbeitung
ausgedruckt werden. 
Intern mu"s dann daf"ur gesorgt werden, da"s sich
Eingabezweig, Ausgabezweig und Verarbeitungszweig
so weit synchronisieren,
da"s die Pufferbereiche weder unter- noch "uberlaufen;
der Verarbeitungszweig soll aber davon nichts merken m"ussen.

Wir beschreiben im Folgenden nur die Eingabeseite;
der Ausgabezweig geht ganz analog.

Als Eingabepuffer verwenden wir im Beispiel einen Ringpuffer
beliebiger aber fester L"ange (wir k"onnen hier noch nicht
die Existenz einer dynamischen Speicherverwaltung voraussetzen).
Die Realisierung ist typisch, 
aber vielleicht etwas sorgsamer als "ublich;
Systemprogramme sind schwer zu testen und 
sollten daher im Prinzip verifizierbar sein.
Die Zugriffsroutinen d"urfen nur aufgerufen werden,
wenn ihre angegebenen Vorbedingungen erf"ullt sind.
Zu beachten ist, da"s ein Pufferelement,
w"ahrend es gerade gef"ullt wird,
weder als belegt noch als frei gez"ahlt wird.

\begin{verbatim}

IMPLEMENTATION MODULE KartenTreiber;

FROM Hardware IMPORT Status, StartCR, CRStatus;

CONST 
  imin =  1;              (* minimaler Pufferindex *)
  imax = 10;              (* maximaler Pufferindex *)
  nmax = imax - imin + 1; (* maximale Elementzahl *)

TYPE
  anzahl = [0 .. nmax];
  index  = [imin .. imax];
  buffer = ARRAY index OF KartenBild;

VAR
  Ring: buffer;
  head, tail: index;
  belegt, frei: anzahl;

\end{verbatim}

\begin{verbatim}
PROCEDURE advance(VAR i: index);   
(* weiterzaehlen im Ring *)
(* PRE  imin <= i  <= imax *)
(* POST imin <= i' <= imax *)
BEGIN IF i = imax THEN i := imin
      ELSE i := i + 1 END;
END advance;

PROCEDURE gibElement(VAR y: KartenBild);
(* PRE belegt > 0 *)
BEGIN y := Ring[head];
      advance(head);
      belegt := belegt - 1;
      frei := frei + 1;
END gibElement;

PROCEDURE belegePlatz(VAR n: index);
(* PRE frei > 0 *)
BEGIN advance(tail); n := tail;
      frei := frei - 1;
END belegePlatz;

PROCEDURE StartLeser;
(* PRE frei > 0 *)
VAR Platz: index;
BEGIN belegePlatz(Platz);
      StartCR(Ring[Platz]);
END StartLeser;

\end{verbatim}

\begin{verbatim}
PROCEDURE KarteDa; 
(* dies ist der Kern der Interruptroutine *)
BEGIN IF CRStatus = ready THEN 
      (* Transport erfolgreich *)
        belegt := belegt + 1;
        IF frei > 0 THEN StartLeser;
        END;
      ELSE Fehlerbehandlung;
      END;
      Dispatcher;
END KarteDa;

\end{verbatim}

F"ur den Fall eines erfolgreichen Transports wird also das Lesen der
n"achsten Karte angesto"sen, au"ser wenn der Puffer voll ist; dann
bleibt der Leser stehen und mu"s sp"ater wieder gestartet werden,
wenn wieder hinreichend Platz ist. Im Falle einer Fehler-R"uckmeldung
mu"s der Operateur verst"andigt werden und eingreifen, damit der
Transport wiederholt werden kann.

Die hier angesprochene Routine {\tt Dispatcher} k"onnte daf"ur sorgen,
da"s an Stelle des unterbrochenen Benutzerprogramms ein anderes
wom"oglich wichtigeres anschlie"send fortgesetzt wird
(preemptive multitasking); sie k"onnte auch im Rahmen der
Interrupt-Behandlung stehen. Lokal hat sie keine Auswirkung.

\begin{verbatim}

PROCEDURE LadeKarte(VAR Ort: KartenBild);
(* erste Fassung, zu verbessern! *)
BEGIN IF belegt = 0 THEN (* Puffer leer *)
        IF CRStatus = ready THEN StartLeser (* Leser steht *)
        END;
        REPEAT Dispatcher;
        UNTIL belegt > 0;
      END;
      gibElement(Ort);
END LadeKarte;

BEGIN (*Initialisierung *)
  head := imax; tail := imax;
  belegt := 0; frei := nmax;
  StartLeser;
END KartenTreiber.

\end{verbatim}

Hier wird die Routine {\tt Dispatcher} deshalb angesprochen,
um an Stelle eines dynamischen Wartens eventuell zwischenzeitlich
ein anderes Benutzerprogramm zu bedienen
(cooperative multitasking).

Die angegebene L"osung ist leider noch fehlerhaft:
sowohl die Routine {\tt LadeKarte} wie auch die Interruptroutine
greifen auf dieselben Daten zu, und wann die Interruptroutine
aufgerufen wird, ist nicht vorhersehbar; es kann also zu b"osen
Synchronisationsfehlern kommen.
Um dies zu verhindern, k"onnte man daran denken,
beim Betreten von {\tt LadeKarte} das Aufrufen der Interruptroutine 
durch einen geeigneten Mechanismus zu sperren 
und es beim Verlassen wieder zuzulassen.
Das gen"ugt allerdings nicht; auch innerhalb der {\tt REPEAT}-Schleife
mu"s die Sperre jeweils gel"ost und sp"ater
wieder gesetzt werden, denn die Bedingung zum Verlassen der Schleife
kann nur durch die Interruptroutine selbst hergestellt werden.
So ergibt sich die verbesserte Version:

\begin{verbatim}

PROCEDURE LadeKarte(VAR Ort: KartenBild);
(* zweite Fassung mit Sperre *)
BEGIN Sperre;
      IF belegt = 0 THEN (* Puffer leer *)
        IF CRStatus = ready THEN StartLeser (* Leser steht *)
        END;
        REPEAT Freigabe; Dispatcher; Sperre;
        UNTIL belegt > 0;
      END;
      gibElement(Ort);
      Freigabe;
END LadeKarte;

\end{verbatim}

Den Aufruf der Interruptroutine k"onnen wir durch Sperren des
zugeh"origen Interrupts verhindern, falls unser Prozessor daf"ur
und f"ur die anschlie"sende Freigabe einen entsprechenden
Maschinenbefehl zur Verf"ugung stellt.

Wir haben damit hier lokal unser Synchronisationsproblem gel"ost;
f"ur den allgemeineren Fall, da"s ein Programmst"uck 
bzw. seine Daten gegen Zugriff durch beliebige parallel ablaufende
Aktivit"aten, auch ggf. auf anderen Prozessoren, zu sch"utzen sind,
werden wir sp"ater allgemeinere Mechanismen kennenlernen.

%%%%%%%%%%%%%%%%%%%%%%%%%%%%%%%%%%%%%%%%%%%%%%%%%%%%%%%%%%%%%%%%%%%%%%%
\endinput

